\documentclass[12pt]{article}
\usepackage[english]{babel}
\usepackage[utf8x]{inputenc}
\usepackage[colorlinks]{hyperref}
\usepackage{amsmath}
\usepackage{graphicx}
\usepackage[colorinlistoftodos]{todonotes}

\begin{document}
	
	\begin{titlepage}
		
		\newcommand{\HRule}{\rule{\linewidth}{0.5mm}} % Defines a new command for the horizontal lines, change thickness here
		
		\center % Center everything on the page
		
		%----------------------------------------------------------------------------------------
		%	HEADING SECTIONS
		%----------------------------------------------------------------------------------------
		
		\textsc{\LARGE Central Washington University}\\[1.5cm] % Name of your university/college
		\textsc{\Large Introduction to Computer Security}\\[0.5cm] % Major heading such as course name
		\textsc{\large Spring 2019}\\[0.5cm] % Minor heading such as course title
		
		%----------------------------------------------------------------------------------------
		%	TITLE SECTION
		%----------------------------------------------------------------------------------------
		
		\HRule \\[0.4cm]
		{ \huge \bfseries Project 2 Report}\\[0.4cm] % Title of your document
		\HRule \\[1.5cm]
		
		%----------------------------------------------------------------------------------------
		%	AUTHOR SECTION
		%----------------------------------------------------------------------------------------
		
		\begin{minipage}{0.4\textwidth}
			\begin{flushleft} \large
				\emph{Author:}\\
				Hermann \textsc{Yepdjio} % Your name
			\end{flushleft}
		\end{minipage}
		~
		\begin{minipage}{0.4\textwidth}
			\begin{flushright} \large
				\emph{Instructor:} \\
				Dr. Razvan \textsc{Andonie} % Supervisor's Name
			\end{flushright}
		\end{minipage}\\[1cm]
		
		% If you don't want a supervisor, uncomment the two lines below and remove the section above
		%\Large \emph{Author:}\\
		%John \textsc{Smith}\\[3cm] % Your name
		
		%----------------------------------------------------------------------------------------
		%	DATE SECTION
		%----------------------------------------------------------------------------------------
		
		{\large \today}\\ % Date, change the \today to a set date if you want to be precise
		
		%----------------------------------------------------------------------------------------
		%	LOGO SECTION
		%----------------------------------------------------------------------------------------
		
		\includegraphics[width=12cm]{CWU-Logo.png}\\[.5cm] % Include a department/university logo - this will require the graphicx package
		
		%----------------------------------------------------------------------------------------
		
		\vfill % Fill the rest of the page with whitespace
		
	\end{titlepage}
	\newpage
	\tableofcontents
	\newpage
	
	
	
	\section{Results}
		\subsection{Part 1}
			Alice's RSA public key  is (N, e) = (33, 3) and her private key is d = 7\\
			a)\\
			
				if Bob encrypts the message M = 19 using Alice's public key, the cipher text C is \\
				C = $M^{e}$ mod N = $19^{3}$ = 6859 = 28 mod 33. {\color{red} C = 28.} \\
				Alice can decrypt C to obtain M by doing the following \\
				M = $C^{d}$ mod N = $28^{7}$ = 13492928512 = 19 mod 33. {\color{red} M = 19}\\\\
			b)\\
			
				If S is the result when Alice  digitally signs the message M = 25, then\\
				S = $M^{d}$ mod N = $25^{7}$ = 6103515625 = 31 mod 33 {\color{red} S = 31}\\\\
				
				If Bob receives M ans S, to verify the signature he just have to unsign S using Alice's public key and see if he obtains M as follow\\
				M = $\{S\}_{alice}$ = $31^{3}$ = 29791 = 25 mod 33 {\color{red} M = 25} 
				
		
		\subsection{Part 2}
			
			Public Key = (18, 30, 7, 26) and n = 47\\
			
			a) Find the private key, assuming m = 6\\\\
				x.6 mod 47 = 18  $\equiv$ x = 3.\\
				x.6 mod 47 = 30  $\equiv$ x = 5.\\
				x.6 mod 47 = 7  $\equiv$ x = 9.\\
				x.6 mod 47 = 26  $\equiv$ x = 20.\\
				
				{\color{red} private key = (3, 5, 9, 20)}\\\\
				
			b) Encryption of M = 1101 (given in binary)\\
			
				18 + 30 + 26 = 74 = {\color{red} 27 mod 47}
				
			\subsection{part 3}
			Output after running the C code:\\
			Point P = (2 , 7) is on the elliptic curve E.\\
			What Alice sent to Bob is : (153 , 36)\\
			What Bob sent to Alice is : (103 , 153)\\
			The shared secret is : (137 , 54)\\
			
				
		
		\section{Observations}
		It was more convenient to solve Part 1 and part 2 of this assignment by hand while it was easier to solve part 3 by writing a program that would do it.
	
	
\end{document}